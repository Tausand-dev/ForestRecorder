%%%%%%%%%%%%%%%%%%%%%%%%%%%%%%%%%%%%%%%%%
% Engineering Calculation Paper
% LaTeX Template
% Version 1.0 (20/1/13)
%
% This template has been downloaded from:
% http://www.LaTeXTemplates.com
%
% Original author:
% Dmitry Volynkin (dim_voly@yahoo.com.au)
%
% License:
% CC BY-NC-SA 3.0 (http://creativecommons.org/licenses/by-nc-sa/3.0/)
%
%%%%%%%%%%%%%%%%%%%%%%%%%%%%%%%%%%%%%%%%%

%----------------------------------------------------------------------------------------
%	PACKAGES AND OTHER DOCUMENT CONFIGURATIONS
%----------------------------------------------------------------------------------------

\documentclass[10pt, letterpaper]{article} % Use A4 paper with a 12pt font size - different paper sizes will require manual recalculation of page margins and border positions

\usepackage{marginnote} % Required for margin notes
\usepackage{wallpaper} % Required to set each page to have a background
\usepackage{lastpage} % Required to print the total number of pages
\usepackage[left = 1.3cm, right = 4.6cm, top = 1.8cm, bottom = 4.0cm, marginparwidth = 3.4cm]{geometry} % Adjust page margins
\usepackage{amsmath} % Required for equation customization
\usepackage{amssymb} % Required to include mathematical symbols
\usepackage{xcolor} % Required to specify colors by name

\usepackage{fancyhdr} % Required to customize headers
\setlength{\headheight}{80pt} % Increase the size of the header to accommodate meta-information
\pagestyle{fancy}\fancyhf{} % Use the custom header specified below
\renewcommand{\headrulewidth}{0pt} % Remove the default horizontal rule under the header

\setlength{\parindent}{0cm} % Remove paragraph indentation
\newcommand{\tab}{\hspace*{2em}} % Defines a new command for some horizontal space

\newcommand\BackgroundStructure{ % Command to specify the background of each page
\setlength{\unitlength}{1mm} % Set the unit length to millimeters

\setlength\fboxsep{0mm} % Adjusts the distance between the frameboxes and the borderlines
\setlength\fboxrule{0.5mm} % Increase the thickness of the border line
\put(10, 10){\fcolorbox{black}{orange!10}{\framebox(165, 230){}}} % Main content box
\put(175, 10){\fcolorbox{black}{orange!25}{\framebox(27, 230){}}} % Margin box
\put(10, 245){\fcolorbox{black}{white!10}{\framebox(192, 20){}}} % Header box
\put(135, 250){\includegraphics[height=10mm, keepaspectratio]{logo}} % Logo box - maximum height/width: 
}

%----------------------------------------------------------------------------------------
%	HEADER INFORMATION
%----------------------------------------------------------------------------------------

\fancyhead[L]{\begin{tabular}{l r | l r} % The header is a table with 4 columns
\textbf{Project} & Forest Recorder & \textbf{Page} & \thepage/\pageref{LastPage} \\ % Project name and page count
\textbf{Version} & FR-001 & \textbf{Updated} & 14/06/2018 \\ % Job number and last updated date
\textbf{Author} & Juan Barbosa & \textbf{Reviewed} &  \\ % Version and reviewed date
& & \textbf{Reviewer} &  \\ % Designer and reviewer
\end{tabular}}

%----------------------------------------------------------------------------------------

\begin{document}

\AddToShipoutPicture{\BackgroundStructure} % Set the background of each page to that specified above in the header information section

%----------------------------------------------------------------------------------------
%	DOCUMENT CONTENT
%----------------------------------------------------------------------------------------

\section{Settings}
	Internal implementation of the arduino code requires the following libraries: \verb|SPI|, \verb|Adafruit_VS1053|, \verb|SdFat|, \verb|RTClibExtended| which are arduino only. The following AVR libraries are used as well: \verb|avr/sleep|, \verb|avr/power|, \verb|avr/wdt|.
	
	\subsection{RTC DS3231}
		Connection with the real time clock is made using the I$^2$C protocol. Wires go as follows:
		\begin{itemize}
			\item \verb|A4 (PC4)| $\rightarrow$ \verb|SDA|
			\item \verb|A5 (PC5)| $\rightarrow$ \verb|SDL|
			\item \verb|D2 (PD2)| $\rightarrow$ \verb|SWQ|
		\end{itemize}
	
		The last of the connections is used for the alarm wake up, since the arduino is on \verb|POWER_DOWN_MODE| signal needs to be set on one of the external interrupt pins, in this case \verb|INT0|.
	
	\subsection{VS1053 breakout}
		The board uses SPI communication, connections are port to port. There are 3 chip select cables, one for the board (\verb|CS|), one for the board memory (\verb|XDCS|) and the SD select (\verb|SDCS|).
		\begin{itemize}
			\item \verb|D5 (PD5)| $\rightarrow$ \verb|DREQ|
			\item \verb|D8 (PB0)| $\rightarrow$ \verb|XDCS|
			\item \verb|D9 (PB1)| $\rightarrow$ \verb|RST|
			\item \verb|D10 (PB2)| $\rightarrow$ \verb|CS|
			\item \verb|D11 (PB3)| $\rightarrow$ \verb|MOSI|
			\item \verb|D12 (PB4)| $\rightarrow$ \verb|MISO|
			\item \verb|D13 (PB5)| $\rightarrow$ \verb|SCLK|
		\end{itemize}
	
	\subsection{SD}
		Connection is made using the breakout SPI ports which were listed before. Chip Select requires a wire from \verb|D4 (PD4)| to \verb|SDCS|.\\
		
		MicroSD card \textbf{must} be \verb|FAT32| formatted.
	
	\subsection{UART}
		% \marginnote{All units are \\ \textbf{[kN, mm]}}
		Communication with the Arduino is made through UART protocol. Settings are set as follows:
		\begin{itemize}
			\item \verb|baudrate = 9600|
			\item \verb|bytesize = EIGHTBITS|
			\item \verb|parity = PARITY_NONE|
			\item \verb|stopbits = STOPBITS_ONE|
		\end{itemize}

\newpage
\section{Scheduling}
	Recording activity is set on a file called \verb|schedule.dat| which has the following structure:
	\begin{center}
		\verb|d1-m1-y1-H1-M1; d2-m2-y2-H2-M2\n|
	\end{center}
	
	Before the semicolon are the starting times, after it, the stop times. Each time descriptor is made as a 2 digit, zero padded number. That is day 4 must be placed as \verb|04|.
	\begin{itemize}
		\item \verb|d|: day
		\item \verb|m|: month
		\item \verb|y|: year
		\item \verb|H|: hour
		\item \verb|M|: minute
	\end{itemize}
	
	Each line \textbf{must} be 32 bytes.

\section{Time setting}
	To set the current time to the RTC, there must be a UART connection active. On a computer as soon as a communication port is open, and connection is establish with the Arduino, this one automatically restarts. There is a 2 second window in which the micro will wait for incoming data from UART, this time starts as soon as the Arduino writes \verb|Connection\r\n|. If no data is received, it will continue to load its recording protocol. If data is received it will start a UART only mode, which only ends with a hardware restart.\\
	
	There are 3 functions implemented: \verb|setTime|, \verb|getTime| and \verb|reset|.
	
	\subsection{setTime (0x00)}
		\begin{itemize}
			\item Message start: \verb|0x00|
			\item Message longitude: 13 bytes
			\item Message: [0x00, 'd', 'd', 'm', 'm', 'y', 'y', 'H', 'H', 'M', 'M', 'S', 'S'] \textbf{(zero padded)}
		\end{itemize}
		For example on ASCII datetime 14/06/2018 16:50:00 looks like this:
		\begin{center}
			[0, '1', '4', '0', '6', '1', '8', '1', '6', '5', '0', '0', '0']
		\end{center}
		Or using hex notation:
		\begin{center}
			[0x00, 0x31, 0x34, 0x30, 0x36, 0x31, 0x38, 0x31, 0x36, 0x35, 0x30, 0x30, 0x30]
		\end{center}
	\begin{itemize}
		\item Answer: \verb|getTime|
	\end{itemize}
	\subsection{getTime (0x01)}
		\begin{itemize}
			\item Message start: \verb|0x01|
			\item Answer: ``dd,mm,yy,HH,MM,SS" \textbf{(not zero padded)}
			
			For example datetime 14/06/2018 16:50:00 looks like this:
			\begin{center}
				\verb|"14,6,18,16,50,0\r\n"|
			\end{center}
		\end{itemize}
	\subsection{reset (0x02)}
		\begin{itemize}
			\item Message start: \verb|0x02|
		\end{itemize}
\end{document}